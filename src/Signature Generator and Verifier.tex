\documentclass[11pt]{article}

    \usepackage[breakable]{tcolorbox}
    \usepackage{parskip} % Stop auto-indenting (to mimic markdown behaviour)
    

    % Basic figure setup, for now with no caption control since it's done
    % automatically by Pandoc (which extracts ![](path) syntax from Markdown).
    \usepackage{graphicx}
    % Keep aspect ratio if custom image width or height is specified
    \setkeys{Gin}{keepaspectratio}
    % Maintain compatibility with old templates. Remove in nbconvert 6.0
    \let\Oldincludegraphics\includegraphics
    % Ensure that by default, figures have no caption (until we provide a
    % proper Figure object with a Caption API and a way to capture that
    % in the conversion process - todo).
    \usepackage{caption}
    \DeclareCaptionFormat{nocaption}{}
    \captionsetup{format=nocaption,aboveskip=0pt,belowskip=0pt}

    \usepackage{float}
    \floatplacement{figure}{H} % forces figures to be placed at the correct location
    \usepackage{xcolor} % Allow colors to be defined
    \usepackage{enumerate} % Needed for markdown enumerations to work
    \usepackage{geometry} % Used to adjust the document margins
    \usepackage{amsmath} % Equations
    \usepackage{amssymb} % Equations
    \usepackage{textcomp} % defines textquotesingle
    % Hack from http://tex.stackexchange.com/a/47451/13684:
    \AtBeginDocument{%
        \def\PYZsq{\textquotesingle}% Upright quotes in Pygmentized code
    }
    \usepackage{upquote} % Upright quotes for verbatim code
    \usepackage{eurosym} % defines \euro

    \usepackage{iftex}
    \ifPDFTeX
        \usepackage[T1]{fontenc}
        \IfFileExists{alphabeta.sty}{
              \usepackage{alphabeta}
          }{
              \usepackage[mathletters]{ucs}
              \usepackage[utf8x]{inputenc}
          }
    \else
        \usepackage{fontspec}
        \usepackage{unicode-math}
    \fi

    \usepackage{fancyvrb} % verbatim replacement that allows latex
    \usepackage{grffile} % extends the file name processing of package graphics
                         % to support a larger range
    \makeatletter % fix for old versions of grffile with XeLaTeX
    \@ifpackagelater{grffile}{2019/11/01}
    {
      % Do nothing on new versions
    }
    {
      \def\Gread@@xetex#1{%
        \IfFileExists{"\Gin@base".bb}%
        {\Gread@eps{\Gin@base.bb}}%
        {\Gread@@xetex@aux#1}%
      }
    }
    \makeatother
    \usepackage[Export]{adjustbox} % Used to constrain images to a maximum size
    \adjustboxset{max size={0.9\linewidth}{0.9\paperheight}}

    % The hyperref package gives us a pdf with properly built
    % internal navigation ('pdf bookmarks' for the table of contents,
    % internal cross-reference links, web links for URLs, etc.)
    \usepackage{hyperref}
    % The default LaTeX title has an obnoxious amount of whitespace. By default,
    % titling removes some of it. It also provides customization options.
    \usepackage{titling}
    \usepackage{longtable} % longtable support required by pandoc >1.10
    \usepackage{booktabs}  % table support for pandoc > 1.12.2
    \usepackage{array}     % table support for pandoc >= 2.11.3
    \usepackage{calc}      % table minipage width calculation for pandoc >= 2.11.1
    \usepackage[inline]{enumitem} % IRkernel/repr support (it uses the enumerate* environment)
    \usepackage[normalem]{ulem} % ulem is needed to support strikethroughs (\sout)
                                % normalem makes italics be italics, not underlines
    \usepackage{soul}      % strikethrough (\st) support for pandoc >= 3.0.0
    \usepackage{mathrsfs}
    

    
    % Colors for the hyperref package
    \definecolor{urlcolor}{rgb}{0,.145,.698}
    \definecolor{linkcolor}{rgb}{.71,0.21,0.01}
    \definecolor{citecolor}{rgb}{.12,.54,.11}

    % ANSI colors
    \definecolor{ansi-black}{HTML}{3E424D}
    \definecolor{ansi-black-intense}{HTML}{282C36}
    \definecolor{ansi-red}{HTML}{E75C58}
    \definecolor{ansi-red-intense}{HTML}{B22B31}
    \definecolor{ansi-green}{HTML}{00A250}
    \definecolor{ansi-green-intense}{HTML}{007427}
    \definecolor{ansi-yellow}{HTML}{DDB62B}
    \definecolor{ansi-yellow-intense}{HTML}{B27D12}
    \definecolor{ansi-blue}{HTML}{208FFB}
    \definecolor{ansi-blue-intense}{HTML}{0065CA}
    \definecolor{ansi-magenta}{HTML}{D160C4}
    \definecolor{ansi-magenta-intense}{HTML}{A03196}
    \definecolor{ansi-cyan}{HTML}{60C6C8}
    \definecolor{ansi-cyan-intense}{HTML}{258F8F}
    \definecolor{ansi-white}{HTML}{C5C1B4}
    \definecolor{ansi-white-intense}{HTML}{A1A6B2}
    \definecolor{ansi-default-inverse-fg}{HTML}{FFFFFF}
    \definecolor{ansi-default-inverse-bg}{HTML}{000000}

    % common color for the border for error outputs.
    \definecolor{outerrorbackground}{HTML}{FFDFDF}

    % commands and environments needed by pandoc snippets
    % extracted from the output of `pandoc -s`
    \providecommand{\tightlist}{%
      \setlength{\itemsep}{0pt}\setlength{\parskip}{0pt}}
    \DefineVerbatimEnvironment{Highlighting}{Verbatim}{commandchars=\\\{\}}
    % Add ',fontsize=\small' for more characters per line
    \newenvironment{Shaded}{}{}
    \newcommand{\KeywordTok}[1]{\textcolor[rgb]{0.00,0.44,0.13}{\textbf{{#1}}}}
    \newcommand{\DataTypeTok}[1]{\textcolor[rgb]{0.56,0.13,0.00}{{#1}}}
    \newcommand{\DecValTok}[1]{\textcolor[rgb]{0.25,0.63,0.44}{{#1}}}
    \newcommand{\BaseNTok}[1]{\textcolor[rgb]{0.25,0.63,0.44}{{#1}}}
    \newcommand{\FloatTok}[1]{\textcolor[rgb]{0.25,0.63,0.44}{{#1}}}
    \newcommand{\CharTok}[1]{\textcolor[rgb]{0.25,0.44,0.63}{{#1}}}
    \newcommand{\StringTok}[1]{\textcolor[rgb]{0.25,0.44,0.63}{{#1}}}
    \newcommand{\CommentTok}[1]{\textcolor[rgb]{0.38,0.63,0.69}{\textit{{#1}}}}
    \newcommand{\OtherTok}[1]{\textcolor[rgb]{0.00,0.44,0.13}{{#1}}}
    \newcommand{\AlertTok}[1]{\textcolor[rgb]{1.00,0.00,0.00}{\textbf{{#1}}}}
    \newcommand{\FunctionTok}[1]{\textcolor[rgb]{0.02,0.16,0.49}{{#1}}}
    \newcommand{\RegionMarkerTok}[1]{{#1}}
    \newcommand{\ErrorTok}[1]{\textcolor[rgb]{1.00,0.00,0.00}{\textbf{{#1}}}}
    \newcommand{\NormalTok}[1]{{#1}}

    % Additional commands for more recent versions of Pandoc
    \newcommand{\ConstantTok}[1]{\textcolor[rgb]{0.53,0.00,0.00}{{#1}}}
    \newcommand{\SpecialCharTok}[1]{\textcolor[rgb]{0.25,0.44,0.63}{{#1}}}
    \newcommand{\VerbatimStringTok}[1]{\textcolor[rgb]{0.25,0.44,0.63}{{#1}}}
    \newcommand{\SpecialStringTok}[1]{\textcolor[rgb]{0.73,0.40,0.53}{{#1}}}
    \newcommand{\ImportTok}[1]{{#1}}
    \newcommand{\DocumentationTok}[1]{\textcolor[rgb]{0.73,0.13,0.13}{\textit{{#1}}}}
    \newcommand{\AnnotationTok}[1]{\textcolor[rgb]{0.38,0.63,0.69}{\textbf{\textit{{#1}}}}}
    \newcommand{\CommentVarTok}[1]{\textcolor[rgb]{0.38,0.63,0.69}{\textbf{\textit{{#1}}}}}
    \newcommand{\VariableTok}[1]{\textcolor[rgb]{0.10,0.09,0.49}{{#1}}}
    \newcommand{\ControlFlowTok}[1]{\textcolor[rgb]{0.00,0.44,0.13}{\textbf{{#1}}}}
    \newcommand{\OperatorTok}[1]{\textcolor[rgb]{0.40,0.40,0.40}{{#1}}}
    \newcommand{\BuiltInTok}[1]{{#1}}
    \newcommand{\ExtensionTok}[1]{{#1}}
    \newcommand{\PreprocessorTok}[1]{\textcolor[rgb]{0.74,0.48,0.00}{{#1}}}
    \newcommand{\AttributeTok}[1]{\textcolor[rgb]{0.49,0.56,0.16}{{#1}}}
    \newcommand{\InformationTok}[1]{\textcolor[rgb]{0.38,0.63,0.69}{\textbf{\textit{{#1}}}}}
    \newcommand{\WarningTok}[1]{\textcolor[rgb]{0.38,0.63,0.69}{\textbf{\textit{{#1}}}}}


    % Define a nice break command that doesn't care if a line doesn't already
    % exist.
    \def\br{\hspace*{\fill} \\* }
    % Math Jax compatibility definitions
    \def\gt{>}
    \def\lt{<}
    \let\Oldtex\TeX
    \let\Oldlatex\LaTeX
    \renewcommand{\TeX}{\textrm{\Oldtex}}
    \renewcommand{\LaTeX}{\textrm{\Oldlatex}}
    % Document parameters
    % Document title
    \title{Signature Generator and Verifier\\[1.5em]}
    \author{
        Lucas Rocha \\ Universidade de Brasília \\ 211055325
        \and
        Gabriela de Oliveira \\ Universidade de Brasília \\ 211055254
        \and
        Eduardo Marciano \\ Universidade de Brasília \\ 211055227
    }
    \date{}
    
    
    
    
    
    
    
% Pygments definitions
\makeatletter
\def\PY@reset{\let\PY@it=\relax \let\PY@bf=\relax%
    \let\PY@ul=\relax \let\PY@tc=\relax%
    \let\PY@bc=\relax \let\PY@ff=\relax}
\def\PY@tok#1{\csname PY@tok@#1\endcsname}
\def\PY@toks#1+{\ifx\relax#1\empty\else%
    \PY@tok{#1}\expandafter\PY@toks\fi}
\def\PY@do#1{\PY@bc{\PY@tc{\PY@ul{%
    \PY@it{\PY@bf{\PY@ff{#1}}}}}}}
\def\PY#1#2{\PY@reset\PY@toks#1+\relax+\PY@do{#2}}

\@namedef{PY@tok@w}{\def\PY@tc##1{\textcolor[rgb]{0.73,0.73,0.73}{##1}}}
\@namedef{PY@tok@c}{\let\PY@it=\textit\def\PY@tc##1{\textcolor[rgb]{0.24,0.48,0.48}{##1}}}
\@namedef{PY@tok@cp}{\def\PY@tc##1{\textcolor[rgb]{0.61,0.40,0.00}{##1}}}
\@namedef{PY@tok@k}{\let\PY@bf=\textbf\def\PY@tc##1{\textcolor[rgb]{0.00,0.50,0.00}{##1}}}
\@namedef{PY@tok@kp}{\def\PY@tc##1{\textcolor[rgb]{0.00,0.50,0.00}{##1}}}
\@namedef{PY@tok@kt}{\def\PY@tc##1{\textcolor[rgb]{0.69,0.00,0.25}{##1}}}
\@namedef{PY@tok@o}{\def\PY@tc##1{\textcolor[rgb]{0.40,0.40,0.40}{##1}}}
\@namedef{PY@tok@ow}{\let\PY@bf=\textbf\def\PY@tc##1{\textcolor[rgb]{0.67,0.13,1.00}{##1}}}
\@namedef{PY@tok@nb}{\def\PY@tc##1{\textcolor[rgb]{0.00,0.50,0.00}{##1}}}
\@namedef{PY@tok@nf}{\def\PY@tc##1{\textcolor[rgb]{0.00,0.00,1.00}{##1}}}
\@namedef{PY@tok@nc}{\let\PY@bf=\textbf\def\PY@tc##1{\textcolor[rgb]{0.00,0.00,1.00}{##1}}}
\@namedef{PY@tok@nn}{\let\PY@bf=\textbf\def\PY@tc##1{\textcolor[rgb]{0.00,0.00,1.00}{##1}}}
\@namedef{PY@tok@ne}{\let\PY@bf=\textbf\def\PY@tc##1{\textcolor[rgb]{0.80,0.25,0.22}{##1}}}
\@namedef{PY@tok@nv}{\def\PY@tc##1{\textcolor[rgb]{0.10,0.09,0.49}{##1}}}
\@namedef{PY@tok@no}{\def\PY@tc##1{\textcolor[rgb]{0.53,0.00,0.00}{##1}}}
\@namedef{PY@tok@nl}{\def\PY@tc##1{\textcolor[rgb]{0.46,0.46,0.00}{##1}}}
\@namedef{PY@tok@ni}{\let\PY@bf=\textbf\def\PY@tc##1{\textcolor[rgb]{0.44,0.44,0.44}{##1}}}
\@namedef{PY@tok@na}{\def\PY@tc##1{\textcolor[rgb]{0.41,0.47,0.13}{##1}}}
\@namedef{PY@tok@nt}{\let\PY@bf=\textbf\def\PY@tc##1{\textcolor[rgb]{0.00,0.50,0.00}{##1}}}
\@namedef{PY@tok@nd}{\def\PY@tc##1{\textcolor[rgb]{0.67,0.13,1.00}{##1}}}
\@namedef{PY@tok@s}{\def\PY@tc##1{\textcolor[rgb]{0.73,0.13,0.13}{##1}}}
\@namedef{PY@tok@sd}{\let\PY@it=\textit\def\PY@tc##1{\textcolor[rgb]{0.73,0.13,0.13}{##1}}}
\@namedef{PY@tok@si}{\let\PY@bf=\textbf\def\PY@tc##1{\textcolor[rgb]{0.64,0.35,0.47}{##1}}}
\@namedef{PY@tok@se}{\let\PY@bf=\textbf\def\PY@tc##1{\textcolor[rgb]{0.67,0.36,0.12}{##1}}}
\@namedef{PY@tok@sr}{\def\PY@tc##1{\textcolor[rgb]{0.64,0.35,0.47}{##1}}}
\@namedef{PY@tok@ss}{\def\PY@tc##1{\textcolor[rgb]{0.10,0.09,0.49}{##1}}}
\@namedef{PY@tok@sx}{\def\PY@tc##1{\textcolor[rgb]{0.00,0.50,0.00}{##1}}}
\@namedef{PY@tok@m}{\def\PY@tc##1{\textcolor[rgb]{0.40,0.40,0.40}{##1}}}
\@namedef{PY@tok@gh}{\let\PY@bf=\textbf\def\PY@tc##1{\textcolor[rgb]{0.00,0.00,0.50}{##1}}}
\@namedef{PY@tok@gu}{\let\PY@bf=\textbf\def\PY@tc##1{\textcolor[rgb]{0.50,0.00,0.50}{##1}}}
\@namedef{PY@tok@gd}{\def\PY@tc##1{\textcolor[rgb]{0.63,0.00,0.00}{##1}}}
\@namedef{PY@tok@gi}{\def\PY@tc##1{\textcolor[rgb]{0.00,0.52,0.00}{##1}}}
\@namedef{PY@tok@gr}{\def\PY@tc##1{\textcolor[rgb]{0.89,0.00,0.00}{##1}}}
\@namedef{PY@tok@ge}{\let\PY@it=\textit}
\@namedef{PY@tok@gs}{\let\PY@bf=\textbf}
\@namedef{PY@tok@ges}{\let\PY@bf=\textbf\let\PY@it=\textit}
\@namedef{PY@tok@gp}{\let\PY@bf=\textbf\def\PY@tc##1{\textcolor[rgb]{0.00,0.00,0.50}{##1}}}
\@namedef{PY@tok@go}{\def\PY@tc##1{\textcolor[rgb]{0.44,0.44,0.44}{##1}}}
\@namedef{PY@tok@gt}{\def\PY@tc##1{\textcolor[rgb]{0.00,0.27,0.87}{##1}}}
\@namedef{PY@tok@err}{\def\PY@bc##1{{\setlength{\fboxsep}{\string -\fboxrule}\fcolorbox[rgb]{1.00,0.00,0.00}{1,1,1}{\strut ##1}}}}
\@namedef{PY@tok@kc}{\let\PY@bf=\textbf\def\PY@tc##1{\textcolor[rgb]{0.00,0.50,0.00}{##1}}}
\@namedef{PY@tok@kd}{\let\PY@bf=\textbf\def\PY@tc##1{\textcolor[rgb]{0.00,0.50,0.00}{##1}}}
\@namedef{PY@tok@kn}{\let\PY@bf=\textbf\def\PY@tc##1{\textcolor[rgb]{0.00,0.50,0.00}{##1}}}
\@namedef{PY@tok@kr}{\let\PY@bf=\textbf\def\PY@tc##1{\textcolor[rgb]{0.00,0.50,0.00}{##1}}}
\@namedef{PY@tok@bp}{\def\PY@tc##1{\textcolor[rgb]{0.00,0.50,0.00}{##1}}}
\@namedef{PY@tok@fm}{\def\PY@tc##1{\textcolor[rgb]{0.00,0.00,1.00}{##1}}}
\@namedef{PY@tok@vc}{\def\PY@tc##1{\textcolor[rgb]{0.10,0.09,0.49}{##1}}}
\@namedef{PY@tok@vg}{\def\PY@tc##1{\textcolor[rgb]{0.10,0.09,0.49}{##1}}}
\@namedef{PY@tok@vi}{\def\PY@tc##1{\textcolor[rgb]{0.10,0.09,0.49}{##1}}}
\@namedef{PY@tok@vm}{\def\PY@tc##1{\textcolor[rgb]{0.10,0.09,0.49}{##1}}}
\@namedef{PY@tok@sa}{\def\PY@tc##1{\textcolor[rgb]{0.73,0.13,0.13}{##1}}}
\@namedef{PY@tok@sb}{\def\PY@tc##1{\textcolor[rgb]{0.73,0.13,0.13}{##1}}}
\@namedef{PY@tok@sc}{\def\PY@tc##1{\textcolor[rgb]{0.73,0.13,0.13}{##1}}}
\@namedef{PY@tok@dl}{\def\PY@tc##1{\textcolor[rgb]{0.73,0.13,0.13}{##1}}}
\@namedef{PY@tok@s2}{\def\PY@tc##1{\textcolor[rgb]{0.73,0.13,0.13}{##1}}}
\@namedef{PY@tok@sh}{\def\PY@tc##1{\textcolor[rgb]{0.73,0.13,0.13}{##1}}}
\@namedef{PY@tok@s1}{\def\PY@tc##1{\textcolor[rgb]{0.73,0.13,0.13}{##1}}}
\@namedef{PY@tok@mb}{\def\PY@tc##1{\textcolor[rgb]{0.40,0.40,0.40}{##1}}}
\@namedef{PY@tok@mf}{\def\PY@tc##1{\textcolor[rgb]{0.40,0.40,0.40}{##1}}}
\@namedef{PY@tok@mh}{\def\PY@tc##1{\textcolor[rgb]{0.40,0.40,0.40}{##1}}}
\@namedef{PY@tok@mi}{\def\PY@tc##1{\textcolor[rgb]{0.40,0.40,0.40}{##1}}}
\@namedef{PY@tok@il}{\def\PY@tc##1{\textcolor[rgb]{0.40,0.40,0.40}{##1}}}
\@namedef{PY@tok@mo}{\def\PY@tc##1{\textcolor[rgb]{0.40,0.40,0.40}{##1}}}
\@namedef{PY@tok@ch}{\let\PY@it=\textit\def\PY@tc##1{\textcolor[rgb]{0.24,0.48,0.48}{##1}}}
\@namedef{PY@tok@cm}{\let\PY@it=\textit\def\PY@tc##1{\textcolor[rgb]{0.24,0.48,0.48}{##1}}}
\@namedef{PY@tok@cpf}{\let\PY@it=\textit\def\PY@tc##1{\textcolor[rgb]{0.24,0.48,0.48}{##1}}}
\@namedef{PY@tok@c1}{\let\PY@it=\textit\def\PY@tc##1{\textcolor[rgb]{0.24,0.48,0.48}{##1}}}
\@namedef{PY@tok@cs}{\let\PY@it=\textit\def\PY@tc##1{\textcolor[rgb]{0.24,0.48,0.48}{##1}}}

\def\PYZbs{\char`\\}
\def\PYZus{\char`\_}
\def\PYZob{\char`\{}
\def\PYZcb{\char`\}}
\def\PYZca{\char`\^}
\def\PYZam{\char`\&}
\def\PYZlt{\char`\<}
\def\PYZgt{\char`\>}
\def\PYZsh{\char`\#}
\def\PYZpc{\char`\%}
\def\PYZdl{\char`\$}
\def\PYZhy{\char`\-}
\def\PYZsq{\char`\'}
\def\PYZdq{\char`\"}
\def\PYZti{\char`\~}
% for compatibility with earlier versions
\def\PYZat{@}
\def\PYZlb{[}
\def\PYZrb{]}
\makeatother


    % For linebreaks inside Verbatim environment from package fancyvrb.
    \makeatletter
        \newbox\Wrappedcontinuationbox
        \newbox\Wrappedvisiblespacebox
        \newcommand*\Wrappedvisiblespace {\textcolor{red}{\textvisiblespace}}
        \newcommand*\Wrappedcontinuationsymbol {\textcolor{red}{\llap{\tiny$\m@th\hookrightarrow$}}}
        \newcommand*\Wrappedcontinuationindent {3ex }
        \newcommand*\Wrappedafterbreak {\kern\Wrappedcontinuationindent\copy\Wrappedcontinuationbox}
        % Take advantage of the already applied Pygments mark-up to insert
        % potential linebreaks for TeX processing.
        %        {, <, #, %, $, ' and ": go to next line.
        %        _, }, ^, &, >, - and ~: stay at end of broken line.
        % Use of \textquotesingle for straight quote.
        \newcommand*\Wrappedbreaksatspecials {%
            \def\PYGZus{\discretionary{\char`\_}{\Wrappedafterbreak}{\char`\_}}%
            \def\PYGZob{\discretionary{}{\Wrappedafterbreak\char`\{}{\char`\{}}%
            \def\PYGZcb{\discretionary{\char`\}}{\Wrappedafterbreak}{\char`\}}}%
            \def\PYGZca{\discretionary{\char`\^}{\Wrappedafterbreak}{\char`\^}}%
            \def\PYGZam{\discretionary{\char`\&}{\Wrappedafterbreak}{\char`\&}}%
            \def\PYGZlt{\discretionary{}{\Wrappedafterbreak\char`\<}{\char`\<}}%
            \def\PYGZgt{\discretionary{\char`\>}{\Wrappedafterbreak}{\char`\>}}%
            \def\PYGZsh{\discretionary{}{\Wrappedafterbreak\char`\#}{\char`\#}}%
            \def\PYGZpc{\discretionary{}{\Wrappedafterbreak\char`\%}{\char`\%}}%
            \def\PYGZdl{\discretionary{}{\Wrappedafterbreak\char`\$}{\char`\$}}%
            \def\PYGZhy{\discretionary{\char`\-}{\Wrappedafterbreak}{\char`\-}}%
            \def\PYGZsq{\discretionary{}{\Wrappedafterbreak\textquotesingle}{\textquotesingle}}%
            \def\PYGZdq{\discretionary{}{\Wrappedafterbreak\char`\"}{\char`\"}}%
            \def\PYGZti{\discretionary{\char`\~}{\Wrappedafterbreak}{\char`\~}}%
        }
        % Some characters . , ; ? ! / are not pygmentized.
        % This macro makes them "active" and they will insert potential linebreaks
        \newcommand*\Wrappedbreaksatpunct {%
            \lccode`\~`\.\lowercase{\def~}{\discretionary{\hbox{\char`\.}}{\Wrappedafterbreak}{\hbox{\char`\.}}}%
            \lccode`\~`\,\lowercase{\def~}{\discretionary{\hbox{\char`\,}}{\Wrappedafterbreak}{\hbox{\char`\,}}}%
            \lccode`\~`\;\lowercase{\def~}{\discretionary{\hbox{\char`\;}}{\Wrappedafterbreak}{\hbox{\char`\;}}}%
            \lccode`\~`\:\lowercase{\def~}{\discretionary{\hbox{\char`\:}}{\Wrappedafterbreak}{\hbox{\char`\:}}}%
            \lccode`\~`\?\lowercase{\def~}{\discretionary{\hbox{\char`\?}}{\Wrappedafterbreak}{\hbox{\char`\?}}}%
            \lccode`\~`\!\lowercase{\def~}{\discretionary{\hbox{\char`\!}}{\Wrappedafterbreak}{\hbox{\char`\!}}}%
            \lccode`\~`\/\lowercase{\def~}{\discretionary{\hbox{\char`\/}}{\Wrappedafterbreak}{\hbox{\char`\/}}}%
            \catcode`\.\active
            \catcode`\,\active
            \catcode`\;\active
            \catcode`\:\active
            \catcode`\?\active
            \catcode`\!\active
            \catcode`\/\active
            \lccode`\~`\~
        }
    \makeatother

    \let\OriginalVerbatim=\Verbatim
    \makeatletter
    \renewcommand{\Verbatim}[1][1]{%
        %\parskip\z@skip
        \sbox\Wrappedcontinuationbox {\Wrappedcontinuationsymbol}%
        \sbox\Wrappedvisiblespacebox {\FV@SetupFont\Wrappedvisiblespace}%
        \def\FancyVerbFormatLine ##1{\hsize\linewidth
            \vtop{\raggedright\hyphenpenalty\z@\exhyphenpenalty\z@
                \doublehyphendemerits\z@\finalhyphendemerits\z@
                \strut ##1\strut}%
        }%
        % If the linebreak is at a space, the latter will be displayed as visible
        % space at end of first line, and a continuation symbol starts next line.
        % Stretch/shrink are however usually zero for typewriter font.
        \def\FV@Space {%
            \nobreak\hskip\z@ plus\fontdimen3\font minus\fontdimen4\font
            \discretionary{\copy\Wrappedvisiblespacebox}{\Wrappedafterbreak}
            {\kern\fontdimen2\font}%
        }%

        % Allow breaks at special characters using \PYG... macros.
        \Wrappedbreaksatspecials
        % Breaks at punctuation characters . , ; ? ! and / need catcode=\active
        \OriginalVerbatim[#1,codes*=\Wrappedbreaksatpunct]%
    }
    \makeatother

    % Exact colors from NB
    \definecolor{incolor}{HTML}{303F9F}
    \definecolor{outcolor}{HTML}{D84315}
    \definecolor{cellborder}{HTML}{CFCFCF}
    \definecolor{cellbackground}{HTML}{F7F7F7}

    % prompt
    \makeatletter
    \newcommand{\boxspacing}{\kern\kvtcb@left@rule\kern\kvtcb@boxsep}
    \makeatother
    \newcommand{\prompt}[4]{
        {\ttfamily\llap{{\color{#2}[#3]:\hspace{3pt}#4}}\vspace{-\baselineskip}}
    }
    

    
    % Prevent overflowing lines due to hard-to-break entities
    \sloppy
    % Setup hyperref package
    \hypersetup{
      breaklinks=true,  % so long urls are correctly broken across lines
      colorlinks=true,
      urlcolor=urlcolor,
      linkcolor=linkcolor,
      citecolor=citecolor,
      }
    % Slightly bigger margins than the latex defaults
    
    \geometry{verbose,tmargin=1in,bmargin=1in,lmargin=1in,rmargin=1in}
    
    

\begin{document}
    
    \maketitle
    
    

    
    \hypertarget{gerador-e-verificador-de-assinaturas-virtuais}{%
\section{Gerador e Verificador de Assinaturas
Virtuais}\label{gerador-e-verificador-de-assinaturas-virtuais}}

Geradores de assinaturas digitais são ferramentas as quais garantem a
autenticidade e/ou a integridade de uma dada mensagem. Para isso,
utiliza-se técnicas de criptografia, normalmente criptografia
assimétrica, onde uma chave privada é empregada para assinar
digitalmente um dado, e o hash da mensagem é incorporado na assinatura
como uma forma de validação do conteúdo.

Já verificadores de assinaturas são responsáveis por confirmar a
autenticidade e/ou a integridade de uma assinatura digital. Considerando
um cenário de criptografia assimétrica como o dado acima, o verificador
utiliza a chave pública correspondente do emissor para decifrar a
mensagem cifrada, caso a mensagem realmente seja do emissor declaro, o
hash deste texto em claro deve ser correspondente ao hash enviado pelo o
emissor.

Assim, neste notebook iremos implementar um gerador e verificador de
assinatura digitais de documentos, utilizando técnicas avançadas em
segurança da computação. Para isso, o projeto consta com três módulos
principais: geração de chaves, encriptação e decriptação, assinatura e
verificação.

O primeiro módulo será responsável pela geração de chaves criptográficas
com base no algoritmo RSA, onde essas chaves serão derivadas de números
primos de 1024 bits. O segundo módulo dedica-se ao processo de
encriptação e decriptação de mensagens utilizando o OAEP (Optimal
Asymmetric Encryption Padding). O terceiro realiza o cálculo do hash da
mensagem, formatação do resultado em BASE64 e sumariza a verificação de
assinaturas digitais com um exemplo prático.

    \hypertarget{gerauxe7uxe3o-de-chaves}{%
\subsection{Geração de Chaves}\label{gerauxe7uxe3o-de-chaves}}

Primeiramente começaremos nosso projeto com a parte mais crucial nos
sistemas modernos de criptografia, a geração das chaves. As chaves são
responsáveis por garantir a confidencialidade e a integridade de
qualquer algoritmo de criptografia moderno, então é necessário uma alta
atenção para se evitar qualquer tipo de padrão ou rastreabilidade em sua
geração. Para isso, usaremos duas bibliotecas que irão nos trazer a
aleatoriedade necessária para a geração das Chaves.

    \begin{tcolorbox}[breakable, size=fbox, boxrule=1pt, pad at break*=1mm,colback=cellbackground, colframe=cellborder]
\prompt{In}{incolor}{1}{\boxspacing}
\begin{Verbatim}[commandchars=\\\{\}]
\PY{k+kn}{from}\PY{+w}{ }\PY{n+nn}{random}\PY{+w}{ }\PY{k+kn}{import} \PY{n}{getrandbits}\PY{p}{,} \PY{n}{randrange}
\end{Verbatim}
\end{tcolorbox}

    O algoritmo que será responsável pela encriptação e decriptação do nosso
projeto será o algoritmo de criptografia assimétrica RSA, amplamente
reconhecido por sua segurança e aplicabilidade em sistemas modernos de
criptografia assimétrica.

Dito isso, por ser assimétrico, iremos gerar duas chaves, uma pública e
uma privada. A chave pública será composta por n=p×q, onde ``p'' e ``q''
são números primos com no mínimo 1024 bits, e por ``e'', o qual será um
número inteiro escolhido que satisfaça a propriedade 1\textless{} ``e''
\textless{} ϕ(n) e sendo coprimo de ϕ(n) = (p−1)×(q−1). Já a chave
privada será representada pelo mesmo n da chave pública e por ``d'',
calculado como ``d''≡ ``e''⁻¹ (mod  ϕ(n)), sendo ``d'' o inverso modular
de ``e'' em relação a ϕ(n).

    Com isto, o primeiro passo será gerar os primos ``q'' e ``p''. Faremos
isso usando o teste de primalidade de Miller--Rabin, o que nos retornará
dois primos diferentes entre si com uma alta probabilidade.

    \begin{tcolorbox}[breakable, size=fbox, boxrule=1pt, pad at break*=1mm,colback=cellbackground, colframe=cellborder]
\prompt{In}{incolor}{2}{\boxspacing}
\begin{Verbatim}[commandchars=\\\{\}]
\PY{k}{def}\PY{+w}{ }\PY{n+nf}{is\PYZus{}prime}\PY{p}{(}\PY{n}{n}\PY{p}{:} \PY{n+nb}{int}\PY{p}{,} \PY{n}{k}\PY{o}{=}\PY{l+m+mi}{128}\PY{p}{)} \PY{o}{\PYZhy{}}\PY{o}{\PYZgt{}} \PY{n+nb}{bool}\PY{p}{:}
\PY{+w}{    }\PY{l+s+sd}{\PYZdq{}\PYZdq{}\PYZdq{}Teste de primalidade usando o algoritmo de Miller\PYZhy{}Rabin.\PYZdq{}\PYZdq{}\PYZdq{}}
    \PY{n}{s} \PY{o}{=} \PY{l+m+mi}{0}
    \PY{n}{r} \PY{o}{=} \PY{n}{n} \PY{o}{\PYZhy{}} \PY{l+m+mi}{1}
    \PY{k}{while} \PY{n}{r} \PY{o}{\PYZam{}} \PY{l+m+mi}{1} \PY{o}{==} \PY{l+m+mi}{0}\PY{p}{:}
        \PY{n}{s} \PY{o}{+}\PY{o}{=} \PY{l+m+mi}{1}
        \PY{n}{r} \PY{o}{/}\PY{o}{/}\PY{o}{=} \PY{l+m+mi}{2}

    \PY{k}{for} \PY{n}{\PYZus{}} \PY{o+ow}{in} \PY{n+nb}{range}\PY{p}{(}\PY{n}{k}\PY{p}{)}\PY{p}{:}
        \PY{n}{a} \PY{o}{=} \PY{n}{randrange}\PY{p}{(}\PY{l+m+mi}{2}\PY{p}{,} \PY{n}{n} \PY{o}{\PYZhy{}} \PY{l+m+mi}{1}\PY{p}{)}
        \PY{n}{x} \PY{o}{=} \PY{n+nb}{pow}\PY{p}{(}\PY{n}{a}\PY{p}{,} \PY{n}{r}\PY{p}{,} \PY{n}{n}\PY{p}{)}
        \PY{k}{if} \PY{n}{x} \PY{o}{!=} \PY{l+m+mi}{1} \PY{o+ow}{and} \PY{n}{x} \PY{o}{!=} \PY{n}{n} \PY{o}{\PYZhy{}} \PY{l+m+mi}{1}\PY{p}{:}
            \PY{n}{j} \PY{o}{=} \PY{l+m+mi}{1}
            \PY{k}{while} \PY{n}{j} \PY{o}{\PYZlt{}} \PY{n}{s} \PY{o+ow}{and} \PY{n}{x} \PY{o}{!=} \PY{n}{n} \PY{o}{\PYZhy{}} \PY{l+m+mi}{1}\PY{p}{:}
                \PY{n}{x} \PY{o}{=} \PY{n+nb}{pow}\PY{p}{(}\PY{n}{x}\PY{p}{,} \PY{l+m+mi}{2}\PY{p}{,} \PY{n}{n}\PY{p}{)}
                \PY{k}{if} \PY{n}{x} \PY{o}{==} \PY{l+m+mi}{1}\PY{p}{:}
                    \PY{k}{return} \PY{k+kc}{False}
                \PY{n}{j} \PY{o}{+}\PY{o}{=} \PY{l+m+mi}{1}
            \PY{k}{if} \PY{n}{x} \PY{o}{!=} \PY{n}{n} \PY{o}{\PYZhy{}} \PY{l+m+mi}{1}\PY{p}{:}
                \PY{k}{return} \PY{k+kc}{False}
    \PY{k}{return} \PY{k+kc}{True}

\PY{k}{def}\PY{+w}{ }\PY{n+nf}{get\PYZus{}prime}\PY{p}{(}\PY{p}{)}\PY{p}{:}
\PY{+w}{    }\PY{l+s+sd}{\PYZdq{}\PYZdq{}\PYZdq{}Gera um número primo de 1024 bits.\PYZdq{}\PYZdq{}\PYZdq{}}
    \PY{k}{while} \PY{k+kc}{True}\PY{p}{:}
        \PY{n}{p} \PY{o}{=} \PY{n}{getrandbits}\PY{p}{(}\PY{l+m+mi}{1024}\PY{p}{)}
        \PY{n}{p} \PY{o}{|}\PY{o}{=} \PY{l+m+mi}{1}
        \PY{n}{p} \PY{o}{|}\PY{o}{=} \PY{l+m+mi}{1} \PY{o}{\PYZlt{}\PYZlt{}} \PY{l+m+mi}{1023}
        \PY{k}{if} \PY{n}{is\PYZus{}prime}\PY{p}{(}\PY{n}{p}\PY{p}{)}\PY{p}{:}
            \PY{k}{return} \PY{n}{p}
        
\PY{n}{p} \PY{o}{=} \PY{n}{get\PYZus{}prime}\PY{p}{(}\PY{p}{)}
\PY{n}{q} \PY{o}{=} \PY{n}{get\PYZus{}prime}\PY{p}{(}\PY{p}{)}

\PY{k}{while} \PY{n}{p} \PY{o}{==} \PY{n}{q}\PY{p}{:}
    \PY{n}{q} \PY{o}{=} \PY{n}{get\PYZus{}prime}\PY{p}{(}\PY{p}{)}
\end{Verbatim}
\end{tcolorbox}

    Com ``p'' e ``q'' gerados, iremos definir algumas funções auxiliares
para calcularmos a operação mdc e os valores de ``n'' e ``phi''.

    \begin{tcolorbox}[breakable, size=fbox, boxrule=1pt, pad at break*=1mm,colback=cellbackground, colframe=cellborder]
\prompt{In}{incolor}{3}{\boxspacing}
\begin{Verbatim}[commandchars=\\\{\}]
\PY{k}{def}\PY{+w}{ }\PY{n+nf}{gcd}\PY{p}{(}\PY{n}{a}\PY{p}{,} \PY{n}{b}\PY{p}{)}\PY{p}{:}
\PY{+w}{    }\PY{l+s+sd}{\PYZdq{}\PYZdq{}\PYZdq{}Calcula o máximo divisor comum (GCD) usando o algoritmo de Euclides.\PYZdq{}\PYZdq{}\PYZdq{}}
    \PY{k}{while} \PY{n}{b}\PY{p}{:}
        \PY{n}{a}\PY{p}{,} \PY{n}{b} \PY{o}{=} \PY{n}{b}\PY{p}{,} \PY{n}{a} \PY{o}{\PYZpc{}} \PY{n}{b}
    \PY{k}{return} \PY{n}{a}

\PY{k}{def}\PY{+w}{ }\PY{n+nf}{get\PYZus{}n}\PY{p}{(}\PY{n}{p}\PY{p}{,} \PY{n}{q}\PY{p}{)}\PY{p}{:}
\PY{+w}{    }\PY{l+s+sd}{\PYZdq{}\PYZdq{}\PYZdq{}Calcula o valor de n = p * q.\PYZdq{}\PYZdq{}\PYZdq{}}
    \PY{k}{return} \PY{n}{p} \PY{o}{*} \PY{n}{q}

\PY{k}{def}\PY{+w}{ }\PY{n+nf}{get\PYZus{}phi}\PY{p}{(}\PY{n}{p}\PY{p}{,} \PY{n}{q}\PY{p}{)}\PY{p}{:}
\PY{+w}{    }\PY{l+s+sd}{\PYZdq{}\PYZdq{}\PYZdq{}Calcula o valor de phi = (p \PYZhy{} 1) * (q \PYZhy{} 1).\PYZdq{}\PYZdq{}\PYZdq{}}
    \PY{k}{return} \PY{p}{(}\PY{n}{p} \PY{o}{\PYZhy{}} \PY{l+m+mi}{1}\PY{p}{)} \PY{o}{*} \PY{p}{(}\PY{n}{q} \PY{o}{\PYZhy{}} \PY{l+m+mi}{1}\PY{p}{)}

\PY{n}{n} \PY{o}{=} \PY{n}{get\PYZus{}n}\PY{p}{(}\PY{n}{p}\PY{p}{,} \PY{n}{q}\PY{p}{)}
\PY{n}{phi} \PY{o}{=} \PY{n}{get\PYZus{}phi}\PY{p}{(}\PY{n}{p}\PY{p}{,} \PY{n}{q}\PY{p}{)}
\end{Verbatim}
\end{tcolorbox}

    Agora, iremos definir duas função que irão encapsular as equações de
geração de ``e'' e ``d''

    \begin{tcolorbox}[breakable, size=fbox, boxrule=1pt, pad at break*=1mm,colback=cellbackground, colframe=cellborder]
\prompt{In}{incolor}{4}{\boxspacing}
\begin{Verbatim}[commandchars=\\\{\}]
\PY{k}{def}\PY{+w}{ }\PY{n+nf}{choose\PYZus{}e}\PY{p}{(}\PY{n}{phi}\PY{p}{)}\PY{p}{:}
\PY{+w}{    }\PY{l+s+sd}{\PYZdq{}\PYZdq{}\PYZdq{}Escolhe um valor de \PYZsq{}e\PYZsq{} que seja coprimo a \PYZsq{}phi\PYZsq{}.\PYZdq{}\PYZdq{}\PYZdq{}}
    \PY{n}{e} \PY{o}{=} \PY{l+m+mi}{2}
    \PY{k}{while} \PY{n}{e} \PY{o}{\PYZlt{}} \PY{n}{phi} \PY{o+ow}{and} \PY{n}{gcd}\PY{p}{(}\PY{n}{e}\PY{p}{,} \PY{n}{phi}\PY{p}{)} \PY{o}{!=} \PY{l+m+mi}{1}\PY{p}{:}
        \PY{n}{e} \PY{o}{+}\PY{o}{=} \PY{l+m+mi}{1}
    \PY{k}{return} \PY{n}{e}

\PY{k}{def}\PY{+w}{ }\PY{n+nf}{get\PYZus{}d}\PY{p}{(}\PY{n}{e}\PY{p}{,} \PY{n}{phi}\PY{p}{)}\PY{p}{:}
\PY{+w}{    }\PY{l+s+sd}{\PYZdq{}\PYZdq{}\PYZdq{}Encotra o inverso modular de \PYZsq{}e\PYZsq{}.\PYZdq{}\PYZdq{}\PYZdq{}}
    \PY{k}{return} \PY{n+nb}{pow}\PY{p}{(}\PY{n}{e}\PY{p}{,} \PY{o}{\PYZhy{}}\PY{l+m+mi}{1}\PY{p}{,} \PY{n}{phi}\PY{p}{)}

\PY{n}{e} \PY{o}{=} \PY{n}{choose\PYZus{}e}\PY{p}{(}\PY{n}{phi}\PY{p}{)}
\PY{n}{d} \PY{o}{=} \PY{n}{get\PYZus{}d}\PY{p}{(}\PY{n}{e}\PY{p}{,} \PY{n}{phi}\PY{p}{)}
\end{Verbatim}
\end{tcolorbox}

    Pronto, com ``n'', ``phi'', ``e'' e ``d'' estamos prontos para gerarmos
nossas chaves com a função generate\_keys();

    \begin{tcolorbox}[breakable, size=fbox, boxrule=1pt, pad at break*=1mm,colback=cellbackground, colframe=cellborder]
\prompt{In}{incolor}{5}{\boxspacing}
\begin{Verbatim}[commandchars=\\\{\}]
\PY{k}{def}\PY{+w}{ }\PY{n+nf}{generate\PYZus{}keys}\PY{p}{(}\PY{n}{n}\PY{p}{,} \PY{n}{e}\PY{p}{,} \PY{n}{d}\PY{p}{)}\PY{p}{:}
\PY{+w}{    }\PY{l+s+sd}{\PYZdq{}\PYZdq{}\PYZdq{}Gera as chaves pública e privada.\PYZdq{}\PYZdq{}\PYZdq{}}
    \PY{n}{public\PYZus{}key} \PY{o}{=} \PY{p}{(}\PY{n}{e}\PY{p}{,} \PY{n}{n}\PY{p}{)}
    \PY{n}{private\PYZus{}key} \PY{o}{=} \PY{p}{(}\PY{n}{d}\PY{p}{,} \PY{n}{n}\PY{p}{)}

    \PY{k}{return} \PY{n}{public\PYZus{}key}\PY{p}{,} \PY{n}{private\PYZus{}key}

\PY{n}{pk}\PY{p}{,} \PY{n}{sk} \PY{o}{=} \PY{n}{generate\PYZus{}keys}\PY{p}{(}\PY{n}{n}\PY{p}{,} \PY{n}{e}\PY{p}{,} \PY{n}{d}\PY{p}{)}

\PY{n+nb}{print}\PY{p}{(}\PY{l+s+sa}{f}\PY{l+s+s2}{\PYZdq{}}\PY{l+s+s2}{Chave Pública: }\PY{l+s+si}{\PYZob{}}\PY{n}{pk}\PY{l+s+si}{\PYZcb{}}\PY{l+s+s2}{.}\PY{l+s+se}{\PYZbs{}n}\PY{l+s+s2}{ Chave privada: }\PY{l+s+si}{\PYZob{}}\PY{n}{sk}\PY{l+s+si}{\PYZcb{}}\PY{l+s+s2}{\PYZdq{}}\PY{p}{)}
\end{Verbatim}
\end{tcolorbox}

    \begin{Verbatim}[commandchars=\\\{\}]
Chave Pública: (3, 1241815923574439537452239892348223837166293932276838398998779
04011383461973814437446103250728416164992254804448189711418036061506800797923293
52452570005848963845403708484997023566736513791836296065532912990717293738924238
48356416547917950829151069948785512228522754730312552267438017700346330232800925
25003514907412111804069231006050069575040505499541536729874264950186742848581399
70058870378763938328798179546119344082458164315176928244635134408458275717354680
75221969554215805282987167582193244606259297035836595218856781374235526641942800
0041942135788014627534377403965635942098639437365027581308343809056631570093).
 Chave privada: (827877282382959691634826594898815891444195954851225599332519360
07588974649209624964068833818944109994836536298793140945357374337867198615529016
35046670565975896935805656664682377824342527890864043688608660478195825949492322
37611031945300552767379965857008152348503153541701511625345133564220155200616833
35675108177126926227444925679230387468529288748751266482342495724553520027197307
76615776678822270271913118586522306197058641383588249177929139341135828533011128
76449295459439413098705337081980045594535239801568446337626940576382964614358593
3678628530049314073340276425136440962686239313055152608298691184155552107, 12418
15923574439537452239892348223837166293932276838398998779040113834619738144374461
03250728416164992254804448189711418036061506800797923293524525700058489638454037
08484997023566736513791836296065532912990717293738924238483564165479179508291510
69948785512228522754730312552267438017700346330232800925250035149074121118040692
31006050069575040505499541536729874264950186742848581399700588703787639383287981
79546119344082458164315176928244635134408458275717354680752219695542158052829871
67582193244606259297035836595218856781374235526641942800004194213578801462753437
7403965635942098639437365027581308343809056631570093)
    \end{Verbatim}

    \hypertarget{encriptauxe7uxe3o-e-decriptauxe7uxe3o}{%
\subsection{Encriptação e
Decriptação}\label{encriptauxe7uxe3o-e-decriptauxe7uxe3o}}

Agora, já em porte da chave pública e a privada, podemos avançar para o
processo de encriptação e decriptação do algoritmo RSA. A encriptação
será realizada com a chave privada (sk), seguindo a fórmula C =
M\^{}``e'' mod n, onde ``M'' é a mensagem original, ``d'' é o expoente
da chave privada e ``n'' é o produto dos números primos gerados. Já para
a decriptação, utilizaremos a chave pública (pk), seguindo essa fórmula:
M = C\^{}``d'', onde ``C'' é o texto cifrado e ``e'' é o expoente da
chave pública.

    \hypertarget{rsa}{%
\subsubsection{RSA}\label{rsa}}

Para encapsular a lógica dessas fórmulas, criaremos uma classe chamada
RSA que irá gerenciar as operações de encriptação e decriptação. No
método de inicialização da classe iremos passar as chaves pública e
privada já definidas.

    \begin{tcolorbox}[breakable, size=fbox, boxrule=1pt, pad at break*=1mm,colback=cellbackground, colframe=cellborder]
\prompt{In}{incolor}{6}{\boxspacing}
\begin{Verbatim}[commandchars=\\\{\}]
\PY{k}{class}\PY{+w}{ }\PY{n+nc}{RSA}\PY{p}{:}
    \PY{k}{def}\PY{+w}{ }\PY{n+nf+fm}{\PYZus{}\PYZus{}init\PYZus{}\PYZus{}}\PY{p}{(}\PY{n+nb+bp}{self}\PY{p}{)}\PY{p}{:}
        \PY{n+nb+bp}{self}\PY{o}{.}\PY{n}{public\PYZus{}key}\PY{p}{,} \PY{n+nb+bp}{self}\PY{o}{.}\PY{n}{private\PYZus{}key} \PY{o}{=} \PY{n}{pk}\PY{p}{,} \PY{n}{sk}

    \PY{k}{def}\PY{+w}{ }\PY{n+nf}{encrypt}\PY{p}{(}\PY{n+nb+bp}{self}\PY{p}{,} \PY{n}{message}\PY{p}{:} \PY{n+nb}{int}\PY{p}{)}\PY{p}{:}
\PY{+w}{        }\PY{l+s+sd}{\PYZdq{}\PYZdq{}\PYZdq{}Criptografa uma mensagem usando a chave privada.\PYZdq{}\PYZdq{}\PYZdq{}}
        \PY{n}{d}\PY{p}{,} \PY{n}{n} \PY{o}{=} \PY{n+nb+bp}{self}\PY{o}{.}\PY{n}{private\PYZus{}key}
        \PY{n}{result} \PY{o}{=} \PY{n+nb}{pow}\PY{p}{(}\PY{n}{message}\PY{p}{,} \PY{n}{d}\PY{p}{,} \PY{n}{n}\PY{p}{)}
        \PY{k}{return} \PY{n+nb}{str}\PY{p}{(}\PY{n}{result}\PY{p}{)}

    \PY{k}{def}\PY{+w}{ }\PY{n+nf}{decrypt}\PY{p}{(}\PY{n+nb+bp}{self}\PY{p}{,} \PY{n}{message}\PY{p}{:} \PY{n+nb}{int}\PY{p}{)}\PY{p}{:}
\PY{+w}{        }\PY{l+s+sd}{\PYZdq{}\PYZdq{}\PYZdq{}Descriptografa uma mensagem usando a chave pública.\PYZdq{}\PYZdq{}\PYZdq{}}
        \PY{n}{e}\PY{p}{,} \PY{n}{n} \PY{o}{=} \PY{n+nb+bp}{self}\PY{o}{.}\PY{n}{public\PYZus{}key}
        \PY{n}{result} \PY{o}{=} \PY{n+nb}{pow}\PY{p}{(}\PY{n}{message}\PY{p}{,} \PY{n}{e}\PY{p}{,} \PY{n}{n}\PY{p}{)}
        \PY{k}{return} \PY{n+nb}{str}\PY{p}{(}\PY{n}{result}\PY{p}{)}
\end{Verbatim}
\end{tcolorbox}

    \hypertarget{oaep}{%
\subsubsection{OAEP}\label{oaep}}

Mas, antes de aplicarmos diretamente o RSA na nossa mensagem a ser
encriptada, primeiramente, iremos utilizar uma técnica de padding pseudo
aleatória, OAEP (Optimal Asymmetric Encryption Padding), com o intuito
de aumentar a segurança da nossa criptografia tornando mais difícil
técnicas baseadas em análises matemáticas e/ou estruturais, as quais
podem possuir alguma efetividade em mensagens com tamanho muito pequeno.

O algoritmo funciona mesclando a mensagem original com um valor pseudo
aleatório r de tamanho de 64 bytes, esse r servirá para gerarmos x e y
que serão a base da nossa nova mensagem com seu padding. Já o algoritmo
de decriptação é determinismo, o qual realiza operações reversas para
obter a mensagem original. Vale ressaltar que esse algoritmo
isoladamente não traz segurança a mensagem, qualquer um que tenha acesso
a cifra será capaz de fazer o processo reverso. No entanto, ele adiciona
uma camada de aleatoriedade, dificultando inferências sobre o conteúdo
da mensagem antes da descriptografia. Por isso, iremos implementá-lo a
seguir encapsulando-o em um classe chamada OAEP.

    \begin{tcolorbox}[breakable, size=fbox, boxrule=1pt, pad at break*=1mm,colback=cellbackground, colframe=cellborder]
\prompt{In}{incolor}{7}{\boxspacing}
\begin{Verbatim}[commandchars=\\\{\}]
\PY{k+kn}{from}\PY{+w}{ }\PY{n+nn}{hashlib}\PY{+w}{ }\PY{k+kn}{import} \PY{n}{sha3\PYZus{}512}
\PY{k+kn}{from}\PY{+w}{ }\PY{n+nn}{os}\PY{+w}{ }\PY{k+kn}{import} \PY{n}{urandom}

\PY{k}{class}\PY{+w}{ }\PY{n+nc}{OAEP}\PY{p}{:}
    \PY{n}{k0} \PY{o}{=} \PY{l+m+mi}{512}
    \PY{n}{k1} \PY{o}{=} \PY{l+m+mi}{256}

    \PY{k}{def}\PY{+w}{ }\PY{n+nf+fm}{\PYZus{}\PYZus{}init\PYZus{}\PYZus{}}\PY{p}{(}\PY{n+nb+bp}{self}\PY{p}{)}\PY{p}{:}
        \PY{n+nb+bp}{self}\PY{o}{.}\PY{n}{x} \PY{o}{=} \PY{k+kc}{None}
        \PY{n+nb+bp}{self}\PY{o}{.}\PY{n}{y} \PY{o}{=} \PY{k+kc}{None}

    \PY{k}{def}\PY{+w}{ }\PY{n+nf}{sha3\PYZus{}512}\PY{p}{(}\PY{n+nb+bp}{self}\PY{p}{,} \PY{n}{data}\PY{p}{:} \PY{n+nb}{bytes}\PY{p}{)} \PY{o}{\PYZhy{}}\PY{o}{\PYZgt{}} \PY{n+nb}{bytes}\PY{p}{:}
\PY{+w}{        }\PY{l+s+sd}{\PYZdq{}\PYZdq{}\PYZdq{}Aplica SHA3\PYZhy{}256 e retorna o hash como bytes.\PYZdq{}\PYZdq{}\PYZdq{}}
        \PY{k}{return} \PY{n}{sha3\PYZus{}512}\PY{p}{(}\PY{n}{data}\PY{p}{)}\PY{o}{.}\PY{n}{digest}\PY{p}{(}\PY{p}{)}

    \PY{k}{def}\PY{+w}{ }\PY{n+nf}{encrypt}\PY{p}{(}\PY{n+nb+bp}{self}\PY{p}{,} \PY{n}{message}\PY{p}{:} \PY{n+nb}{int}\PY{p}{)} \PY{o}{\PYZhy{}}\PY{o}{\PYZgt{}} \PY{n+nb}{int}\PY{p}{:}
\PY{+w}{        }\PY{l+s+sd}{\PYZdq{}\PYZdq{}\PYZdq{}Encriptação do algoritmo OAEP\PYZdq{}\PYZdq{}\PYZdq{}}
        \PY{n}{r} \PY{o}{=} \PY{n}{urandom}\PY{p}{(}\PY{l+m+mi}{64}\PY{p}{)}
        
        \PY{n}{message} \PY{o}{=} \PY{n}{message} \PY{o}{\PYZlt{}\PYZlt{}} \PY{n+nb+bp}{self}\PY{o}{.}\PY{n}{k1}
        
        \PY{n}{x} \PY{o}{=} \PY{n}{message} \PY{o}{\PYZca{}} \PY{n+nb}{int}\PY{o}{.}\PY{n}{from\PYZus{}bytes}\PY{p}{(}\PY{n+nb+bp}{self}\PY{o}{.}\PY{n}{sha3\PYZus{}512}\PY{p}{(}\PY{n}{r}\PY{p}{)}\PY{p}{)}
        
        \PY{n}{y} \PY{o}{=} \PY{n+nb}{int}\PY{o}{.}\PY{n}{from\PYZus{}bytes}\PY{p}{(}\PY{n}{r}\PY{p}{)} \PY{o}{\PYZca{}} \PY{n+nb}{int}\PY{o}{.}\PY{n}{from\PYZus{}bytes}\PY{p}{(}\PY{n+nb+bp}{self}\PY{o}{.}\PY{n}{sha3\PYZus{}512}\PY{p}{(}\PY{n}{x}\PY{o}{.}\PY{n}{to\PYZus{}bytes}\PY{p}{(}\PY{l+m+mi}{128}\PY{p}{)}\PY{p}{)}\PY{p}{)}
        
        \PY{k}{return} \PY{p}{(}\PY{n}{x} \PY{o}{\PYZlt{}\PYZlt{}} \PY{n+nb+bp}{self}\PY{o}{.}\PY{n}{k0}\PY{p}{)} \PY{o}{|} \PY{n}{y}

    \PY{k}{def}\PY{+w}{ }\PY{n+nf}{decrypt}\PY{p}{(}\PY{n+nb+bp}{self}\PY{p}{,} \PY{n}{ciphertext}\PY{p}{:} \PY{n+nb}{int}\PY{p}{)} \PY{o}{\PYZhy{}}\PY{o}{\PYZgt{}} \PY{n+nb}{int}\PY{p}{:}
\PY{+w}{        }\PY{l+s+sd}{\PYZdq{}\PYZdq{}\PYZdq{}Decriptação do algoritmo OAEP\PYZdq{}\PYZdq{}\PYZdq{}}
        
        \PY{n}{y} \PY{o}{=} \PY{n}{ciphertext} \PY{o}{\PYZam{}} \PY{p}{(}\PY{p}{(}\PY{l+m+mi}{1} \PY{o}{\PYZlt{}\PYZlt{}} \PY{n+nb+bp}{self}\PY{o}{.}\PY{n}{k0}\PY{p}{)} \PY{o}{\PYZhy{}} \PY{l+m+mi}{1}\PY{p}{)}
        \PY{n}{x} \PY{o}{=} \PY{n}{ciphertext} \PY{o}{\PYZgt{}\PYZgt{}} \PY{n+nb+bp}{self}\PY{o}{.}\PY{n}{k0}

        \PY{n}{r} \PY{o}{=} \PY{n}{y} \PY{o}{\PYZca{}} \PY{n+nb}{int}\PY{o}{.}\PY{n}{from\PYZus{}bytes}\PY{p}{(}\PY{n+nb+bp}{self}\PY{o}{.}\PY{n}{sha3\PYZus{}512}\PY{p}{(}\PY{n}{x}\PY{o}{.}\PY{n}{to\PYZus{}bytes}\PY{p}{(}\PY{l+m+mi}{128}\PY{p}{)}\PY{p}{)}\PY{p}{)}

        \PY{n}{pm} \PY{o}{=} \PY{n}{x} \PY{o}{\PYZca{}} \PY{n+nb}{int}\PY{o}{.}\PY{n}{from\PYZus{}bytes}\PY{p}{(}\PY{n+nb+bp}{self}\PY{o}{.}\PY{n}{sha3\PYZus{}512}\PY{p}{(}\PY{n}{r}\PY{o}{.}\PY{n}{to\PYZus{}bytes}\PY{p}{(}\PY{n+nb+bp}{self}\PY{o}{.}\PY{n}{k0} \PY{o}{/}\PY{o}{/} \PY{l+m+mi}{8}\PY{p}{)}\PY{p}{)}\PY{p}{)}

        \PY{k}{return} \PY{n}{pm} \PY{o}{\PYZgt{}\PYZgt{}} \PY{n+nb+bp}{self}\PY{o}{.}\PY{n}{k1}
\end{Verbatim}
\end{tcolorbox}

    \hypertarget{assinatura-transmissuxe3o-e-verificauxe7uxe3o-de-assinatura}{%
\subsection{Assinatura, Transmissão e Verificação de
Assinatura}\label{assinatura-transmissuxe3o-e-verificauxe7uxe3o-de-assinatura}}

Agora que já temos ferramentas o suficiente para realizarmos a geração
das chaves, encriptação e decriptação, iremos nos preocupar com
questionamentos fundamentais na nossa aplicação: Como podemos assinar a
mensagem? Como podemos transmitir essa mensagem de uma maneira adequada?
Como podemos verificar nossa assinatura?

    \hypertarget{assinatura}{%
\subsubsection{Assinatura}\label{assinatura}}

Para assinarmos digitalmente nossa mensagem a ser transmitida,
utilizaremos uma função hash, mais especificamente as funções SHA3-256 e
SHA3-512. Uma função de hash é um algoritmo que dado uma entrada de
dados de qualquer tamanho, sua saída será de tamanho fixo, a qual é
chamada de hash. Utilizaremos essa função, pois ela possui a propriedade
de que a computação reversa do hash para a mensagem original é
extremamente custosa computacionalmente, propriedade da
irreversibilidade. Ainda, uma função hash possui uma segunda propriedade
de interesse para nós, a propriedade de resistência à colisão, que
define que o processo de descoberta de uma mensagem que gere um mesmo
hash é extremamente caro computacionalmente.

Assim, a nossa estratégia de assinatura será transmitir nossa mensagem e
o seu hash, porém este hash estará criptografado pelo algoritmo RSA. Com
isso, ao realizar a decriptação utilizando a chave pública do emissor, o
receptor poderá fazer o hash da mensagem recebida e comparar com o hash
decriptado, e, caso eles sejam iguais, o emissor terá a garantia que a
mensagem é realmente do seu emissor, já que apenas ele possui acesso a
sua chave privada capaz de gerar a encriptação correta.

Nesta parte do nosso projeto, utilizaremos a biblioteca ``hashlib'' para
o desenvolvimento da nossa função Hash, sendo a única parte do nosso
projeto que utilizará código de terceiros.

\hypertarget{funuxe7uxe3o-hash}{%
\paragraph{Função Hash}\label{funuxe7uxe3o-hash}}

    \begin{tcolorbox}[breakable, size=fbox, boxrule=1pt, pad at break*=1mm,colback=cellbackground, colframe=cellborder]
\prompt{In}{incolor}{8}{\boxspacing}
\begin{Verbatim}[commandchars=\\\{\}]
\PY{k+kn}{from}\PY{+w}{ }\PY{n+nn}{hashlib}\PY{+w}{ }\PY{k+kn}{import} \PY{n}{sha3\PYZus{}512}\PY{p}{,} \PY{n}{sha3\PYZus{}256}
\PY{k}{class}\PY{+w}{ }\PY{n+nc}{sha3}\PY{p}{:}
  \PY{n+nd}{@staticmethod}
  \PY{k}{def}\PY{+w}{ }\PY{n+nf}{hash\PYZus{}512}\PY{p}{(}\PY{n}{message}\PY{p}{)}\PY{p}{:}
    \PY{k}{if} \PY{n+nb}{type}\PY{p}{(}\PY{n}{message}\PY{p}{)} \PY{o}{==} \PY{n+nb}{int}\PY{p}{:}
      \PY{n}{message} \PY{o}{=} \PY{n+nb}{str}\PY{p}{(}\PY{n}{message}\PY{p}{)}
    \PY{k}{elif} \PY{n+nb}{type}\PY{p}{(}\PY{n}{message}\PY{p}{)} \PY{o}{==} \PY{n+nb}{str}\PY{p}{:}
      \PY{n}{message} \PY{o}{=} \PY{n}{message}\PY{o}{.}\PY{n}{encode}\PY{p}{(}\PY{p}{)}
    \PY{k}{return} \PY{n+nb}{int}\PY{o}{.}\PY{n}{from\PYZus{}bytes}\PY{p}{(}\PY{n}{sha3\PYZus{}512}\PY{p}{(}\PY{n}{message}\PY{p}{)}\PY{o}{.}\PY{n}{digest}\PY{p}{(}\PY{p}{)}\PY{p}{)}

  \PY{n+nd}{@staticmethod}
  \PY{k}{def}\PY{+w}{ }\PY{n+nf}{hash\PYZus{}256}\PY{p}{(}\PY{n}{message}\PY{p}{)}\PY{p}{:}
    \PY{k}{if} \PY{n+nb}{type}\PY{p}{(}\PY{n}{message}\PY{p}{)} \PY{o}{==} \PY{n+nb}{int}\PY{p}{:}
      \PY{n}{message} \PY{o}{=} \PY{n+nb}{str}\PY{p}{(}\PY{n}{message}\PY{p}{)}
    \PY{k}{elif} \PY{n+nb}{type}\PY{p}{(}\PY{n}{message}\PY{p}{)} \PY{o}{==} \PY{n+nb}{str}\PY{p}{:}
      \PY{n}{message} \PY{o}{=} \PY{n}{message}\PY{o}{.}\PY{n}{encode}\PY{p}{(}\PY{p}{)}
    \PY{k}{return} \PY{n+nb}{int}\PY{o}{.}\PY{n}{from\PYZus{}bytes}\PY{p}{(}\PY{n}{sha3\PYZus{}256}\PY{p}{(}\PY{n}{message}\PY{p}{)}\PY{o}{.}\PY{n}{digest}\PY{p}{(}\PY{p}{)}\PY{p}{)}
\end{Verbatim}
\end{tcolorbox}

    \hypertarget{transmissuxe3o}{%
\subsubsection{Transmissão}\label{transmissuxe3o}}

Para a transmissão da nossa cifra e da menssagem, primeiramente, iremos
converter a concatenação de ambas para o formato Base64. Isso será
feito, pois a Base64 é uma forma de codificação que transforma dados
binários em uma sequência de caracteres ASCII, o que normalmente é um
formato mais adequado a transmissão de dados.

Nossa implementação é um modelo simplificado da Base64, mas
suficientemente robusto para o escopo deste projeto.

\hypertarget{base64}{%
\paragraph{BASE64}\label{base64}}

    \begin{tcolorbox}[breakable, size=fbox, boxrule=1pt, pad at break*=1mm,colback=cellbackground, colframe=cellborder]
\prompt{In}{incolor}{9}{\boxspacing}
\begin{Verbatim}[commandchars=\\\{\}]
\PY{k}{class}\PY{+w}{ }\PY{n+nc}{base\PYZus{}convert}\PY{p}{:}

  \PY{n+nd}{@staticmethod}
  \PY{k}{def}\PY{+w}{ }\PY{n+nf}{format}\PY{p}{(}\PY{n}{message}\PY{p}{:} \PY{n+nb}{int}\PY{p}{)}\PY{p}{:}
    \PY{n}{base64\PYZus{}chars} \PY{o}{=} \PY{l+s+s2}{\PYZdq{}}\PY{l+s+s2}{ABCDEFGHIJKLMNOPQRSTUVWXYZabcdefghijklmnopqrstuvwxyz0123456789+/}\PY{l+s+s2}{\PYZdq{}}
    \PY{n}{res} \PY{o}{=} \PY{l+s+s2}{\PYZdq{}}\PY{l+s+s2}{\PYZdq{}}
    \PY{c+c1}{\PYZsh{}Enquanto a mensagem não se tornar 0 realiza o método de divisões sucessivas}
    \PY{k}{while} \PY{n}{message} \PY{o}{\PYZgt{}} \PY{l+m+mi}{0}\PY{p}{:}
      \PY{n}{remainder} \PY{o}{=} \PY{n}{message} \PY{o}{\PYZpc{}} \PY{l+m+mi}{64}
      \PY{n}{res} \PY{o}{=} \PY{n}{base64\PYZus{}chars}\PY{p}{[}\PY{n}{remainder}\PY{p}{]} \PY{o}{+} \PY{n}{res}
      \PY{n}{message} \PY{o}{=} \PY{n}{message} \PY{o}{/}\PY{o}{/} \PY{l+m+mi}{64}
    \PY{k}{return} \PY{n}{res}

  \PY{n+nd}{@staticmethod}
  \PY{k}{def}\PY{+w}{ }\PY{n+nf}{parse}\PY{p}{(}\PY{n}{base64\PYZus{}message}\PY{p}{:} \PY{n+nb}{str}\PY{p}{)}\PY{p}{:}
    \PY{n}{base64\PYZus{}chars} \PY{o}{=} \PY{l+s+s2}{\PYZdq{}}\PY{l+s+s2}{ABCDEFGHIJKLMNOPQRSTUVWXYZabcdefghijklmnopqrstuvwxyz0123456789+/}\PY{l+s+s2}{\PYZdq{}}
    \PY{n}{res} \PY{o}{=} \PY{l+m+mi}{0}
    \PY{c+c1}{\PYZsh{} procura o caracter encontrado na string}
    \PY{k}{for} \PY{n}{char} \PY{o+ow}{in} \PY{n}{base64\PYZus{}message}\PY{p}{:}
      \PY{n}{res} \PY{o}{=} \PY{n}{res} \PY{o}{*} \PY{l+m+mi}{64} \PY{o}{+} \PY{n}{base64\PYZus{}chars}\PY{o}{.}\PY{n}{index}\PY{p}{(}\PY{n}{char}\PY{p}{)}
    \PY{k}{return} \PY{n}{res}
\end{Verbatim}
\end{tcolorbox}

    \hypertarget{verificauxe7uxe3o-com-exemplo-pruxe1tico}{%
\subsubsection{Verificação com Exemplo
Prático}\label{verificauxe7uxe3o-com-exemplo-pruxe1tico}}

Agora, finalmente partiremos para a demonstração do nosso projeto. Nesta
demonstração, geramos o contéudo do nosso plain text utilizando o lorem
ipsum, um gerador randômico de texto. O arquivo em si está disponível no
diretório resources com nome ``input.txt''.

O processo começa com a leitura do texto em formato binário, que é
convertido diretamente em um valor numérico inteiro. Em seguida,
calculamos o hash da mensagem para criar uma representação única e
compacta do texto. Após isso, utilizamos o padding OAEP sobre o hash da
mensagem para adicionar segurança à encriptação. Com o hash preparado,
realizamos a sua criptografia utilizando a chave privada do algoritmo
RSA. Por fim, para realizarmos a transmissão, concatenamos a mensagem ao
hash criptografado e, assim, colocamos o resultado da concatenação em
Base64.

Já o processo de verificação é o reverso no processo de assinatura.
Primeiramente, desfazemos o formato Base64, separamos a mensagem do hash
criptografado e realizamos a decriptação com a chave pública do emissor
e retiramos o padding resultado do OAEP. Assim, ao final dessa série de
processos, temos dois dados fundamentais, a mensagem e o hash do
emissor, caso o hash do emissor seja compatível com o hash da mensagem,
realmente essa mensagem é do emissor declarado, garantindo assim
confiabilidade a origem dessa mensagem, caso não seja, concluímos que ou
o hash ou a mensagem foram interceptadas e trocadas durante a nossa
transmissão, logo, ambas devem ser invalidadas e descartadas.

    \begin{tcolorbox}[breakable, size=fbox, boxrule=1pt, pad at break*=1mm,colback=cellbackground, colframe=cellborder]
\prompt{In}{incolor}{10}{\boxspacing}
\begin{Verbatim}[commandchars=\\\{\}]
\PY{c+c1}{\PYZsh{} Instanciação da classe RSA}
\PY{n}{cy} \PY{o}{=} \PY{n}{RSA}\PY{p}{(}\PY{p}{)}
\PY{n}{oaep} \PY{o}{=} \PY{n}{OAEP}\PY{p}{(}\PY{p}{)}

\PY{n}{f} \PY{o}{=} \PY{n+nb}{open}\PY{p}{(}\PY{l+s+s1}{\PYZsq{}}\PY{l+s+s1}{../resources/input.txt}\PY{l+s+s1}{\PYZsq{}}\PY{p}{,} \PY{l+s+s1}{\PYZsq{}}\PY{l+s+s1}{rb}\PY{l+s+s1}{\PYZsq{}}\PY{p}{)}
\PY{n}{plain\PYZus{}text} \PY{o}{=} \PY{n}{f}\PY{o}{.}\PY{n}{read}\PY{p}{(}\PY{p}{)}
\PY{n}{f}\PY{o}{.}\PY{n}{close}\PY{p}{(}\PY{p}{)}

\PY{n}{plain\PYZus{}text} \PY{o}{=} \PY{n+nb}{str}\PY{p}{(}\PY{n+nb}{int}\PY{o}{.}\PY{n}{from\PYZus{}bytes}\PY{p}{(}\PY{n}{plain\PYZus{}text}\PY{p}{)}\PY{p}{)}

\PY{c+c1}{\PYZsh{} Calculo do hash 256 da mensagem}
\PY{n}{h} \PY{o}{=} \PY{n}{sha3}\PY{o}{.}\PY{n}{hash\PYZus{}256}\PY{p}{(}\PY{n}{plain\PYZus{}text}\PY{p}{)}

\PY{c+c1}{\PYZsh{} Encryptação desse hash}
\PY{n}{e} \PY{o}{=} \PY{n}{oaep}\PY{o}{.}\PY{n}{encrypt}\PY{p}{(}\PY{n}{h}\PY{p}{)}
\PY{n}{e} \PY{o}{=} \PY{n}{cy}\PY{o}{.}\PY{n}{encrypt}\PY{p}{(}\PY{n}{e}\PY{p}{)}

\PY{c+c1}{\PYZsh{} Concatenação entre plain\PYZus{}text e o hash encriptado}
\PY{n}{message\PYZus{}to\PYZus{}encode} \PY{o}{=} \PY{n}{plain\PYZus{}text} \PY{o}{+} \PY{n+nb}{str}\PY{p}{(}\PY{n}{e}\PY{p}{)}

\PY{c+c1}{\PYZsh{} Conversão para a Base64}
\PY{n}{encoded\PYZus{}message} \PY{o}{=} \PY{n}{base\PYZus{}convert}\PY{o}{.}\PY{n}{format}\PY{p}{(}\PY{n+nb}{int}\PY{p}{(}\PY{n}{message\PYZus{}to\PYZus{}encode}\PY{p}{)}\PY{p}{)}

\PY{n+nb}{print}\PY{p}{(}\PY{l+s+s2}{\PYZdq{}}\PY{l+s+s2}{Mensagem codificada em base64:}\PY{l+s+s2}{\PYZdq{}}\PY{p}{,} \PY{n}{encoded\PYZus{}message}\PY{p}{)}

\PY{c+c1}{\PYZsh{} Parse da base64}
\PY{n}{decoded\PYZus{}message} \PY{o}{=} \PY{n+nb}{str}\PY{p}{(}\PY{n}{base\PYZus{}convert}\PY{o}{.}\PY{n}{parse}\PY{p}{(}\PY{n}{encoded\PYZus{}message}\PY{p}{)}\PY{p}{)}

\PY{c+c1}{\PYZsh{} Separação entre o plain\PYZus{}text e o hash encriptado}
\PY{n}{decoded\PYZus{}plain\PYZus{}text} \PY{o}{=} \PY{n}{decoded\PYZus{}message}\PY{p}{[}\PY{p}{:}\PY{n+nb}{len}\PY{p}{(}\PY{n}{plain\PYZus{}text}\PY{p}{)}\PY{p}{]}
\PY{n}{decoded\PYZus{}plain\PYZus{}text\PYZus{}hash} \PY{o}{=} \PY{n}{sha3}\PY{o}{.}\PY{n}{hash\PYZus{}256}\PY{p}{(}\PY{n}{decoded\PYZus{}plain\PYZus{}text}\PY{p}{)}

\PY{n}{encrypted\PYZus{}hash} \PY{o}{=} \PY{n}{decoded\PYZus{}message}\PY{p}{[}\PY{n+nb}{len}\PY{p}{(}\PY{n}{plain\PYZus{}text}\PY{p}{)}\PY{p}{:}\PY{p}{]}

\PY{c+c1}{\PYZsh{} Decryptação do hash encriptado}
\PY{n}{decrypted\PYZus{}hash} \PY{o}{=} \PY{n}{cy}\PY{o}{.}\PY{n}{decrypt}\PY{p}{(}\PY{n+nb}{int}\PY{p}{(}\PY{n}{encrypted\PYZus{}hash}\PY{p}{)}\PY{p}{)}
\PY{n}{decrypted\PYZus{}hash} \PY{o}{=} \PY{n}{oaep}\PY{o}{.}\PY{n}{decrypt}\PY{p}{(}\PY{n+nb}{int}\PY{p}{(}\PY{n}{decrypted\PYZus{}hash}\PY{p}{)}\PY{p}{)}

\PY{c+c1}{\PYZsh{} Comparação entre os hashs para verificação da assinatura}
\PY{n+nb}{print}\PY{p}{(}\PY{l+s+s2}{\PYZdq{}}\PY{l+s+s2}{Hash da mensagem:}\PY{l+s+s2}{\PYZdq{}}\PY{p}{,} \PY{n}{decoded\PYZus{}plain\PYZus{}text\PYZus{}hash}\PY{p}{)}
\PY{n+nb}{print}\PY{p}{(}\PY{l+s+s2}{\PYZdq{}}\PY{l+s+s2}{Hash decodificado:}\PY{l+s+s2}{\PYZdq{}}\PY{p}{,} \PY{n}{decrypted\PYZus{}hash}\PY{p}{)}
\PY{n+nb}{print}\PY{p}{(}\PY{l+s+s2}{\PYZdq{}}\PY{l+s+s2}{Verificação de assinatura:}\PY{l+s+s2}{\PYZdq{}}\PY{p}{,} \PY{n}{decrypted\PYZus{}hash} \PY{o}{==} \PY{n}{decoded\PYZus{}plain\PYZus{}text\PYZus{}hash}\PY{p}{)}
\end{Verbatim}
\end{tcolorbox}

    \begin{Verbatim}[commandchars=\\\{\}]
Mensagem codificada em base64: EOxOvgF3h+Y2r0PvkqnFY5DL+ipBkSGxALH4YdfoARSLVvrfn
KbuGbuKL3w4/sfZjPFLikYQF5ld18/G3pnpXxivosoQvx3KcC9GnQIef0dg7WBtNLXQyS+ikW7qBZ7Zg
2ABzr3pHaEZKm/4rR/uDybjTuZ1fzYk6h0jOgsCQaSHyANDVWRegbPwi6ZEDnZNTnUbKvvLa9mkLM12F
PyyA1lN8RL2Snrdt2P6fKi7HJa+bE5HeQQTJGk2lt+o6RTr9PpBY6kyRvvbZ3spJh25VIyf/lTsSK7pZ
UIAU5q0UQjf27PL//0YTVZ0/CAPfPzBeGIEdzb/0hwMqyhc6Gw8qDbvIBZvjpwtAiBh16mCO3goWA9Qa
tg4ih/LH6dI3V6CNBkU/Fypl7pNFevngSK6NPJHwpWg+gnN9HjSdUXcPJF1mDvU3sphKYloFGvJa4uAx
HO3tOlZ7rKe/I9qBGVV2KsJtI/o32GwL6l28STidGd0CLzXSFXMtb5UqsLZRx4gRBAxiJA0L+rfdNgho
a2WoZYK+d+d/Vks7rBMlHXJZblF67EvQ8LN7bAuACNVv5psnpXNgZzPuzCacDIfUU0rMYGHCL/+wW1ae
06ZjB2Ou6tOBSC65n4P2AD1xUkkKyYAjyneQgeOVXNJwLdtn7OJuQzLEVD0HFFcibsjSapybfkFa5JYE
UhZnyVqVgMxVY+CVWlDYJ4c+qrfMDIx7r3fVT61uNWX7l6oPnVEq4hOd5Zlc/uS/GdQGn2H7u9TsUY7T
1FAr5ozlTYuuTUIKwJZfdifJNJyTY3UYSVthqIpos4TsJP0+SkonlK5qyt8iHngHHeAqjGcIXvQ5VEIo
zZpJ3zG6pG2QQIlxK9bGTOSztTVAiq4+yuV1yY+o2Lmj7E/yM6z5cavR6u2ou6eUURQGc/kzxOFVgqE6
pwpFvqQ9kBt86th5FjSTL/Ro7WTB8h/fnLmTr0uLw5kJp7ShdvdtL0nuMuB1IPWSEmjiniKQbg2D3kOL
vWVDpCw37AZNvjODWsHPhR02P/SMOqeL9I5+8T3/i+f04Fgq2kJOp6xYyMzqVL9q6mX/2n6QJYjYi+LG
C5UUGx4wtGcB3hHd/V1/+hzsr1HgZMH7IvcvHvpRIi7U+KqzzkdTZyCzAE7/LDlZCN5V6hoaLVb4uOPJ
KO2bvLUCKbY9vn6y4qiDHBB2FmP1rewiG0RmyihxygziYjC/zZvkxUnHG/akqwmOcxHoveIYFH+5Xz5O
LXdMZ7189SSrkSWW8TDurpfhnJk0IV6O1+GRx1TOku33f3u0hU3i0xC6sRS5hDAlgZ9y1WlCzJMWip32
esJYSu/0Mgf5d79U/cKzLLf9d8T2ngm25G4SM0GLv9Ux1/Tf5mFF2yLramhksVhzMnrzsZpSkyRuLKsI
RZam0ykcApFCSCc9+AXk9c73Z7B0ywqU2O9dr2XgIL1NW5FZsw8ertbJVRvSmFaRr0w7sDybxO6cJrkY
JAwUGuLx/AAQbkuwJUVO0JdemzqnZIuCP4ksXHJZaZ6tewEAMRwvR8OC5uwnkX9odlnQj3VlAOE1ENzq
mQFgEWD5Rwyyg8G4NCNn6ebom5dPG0kiFtZROLspoz7Dx/3bKTrhYSY2LTvnOY55lLpwU/MO3HYVV/ul
xWRlhxoNkUNd+CpT1/wQZgR3dTVq0izTxGNAuVSSC/6jfwH4ZWTl1VadqndC13Ex8VT6CkTeDnkvngsV
+HN/k69f2r68xtOqzLElwkOxlr+8zhSdCaeeFxTIRM3HQE8oxXkuhcB4b8IxLv0ypvw7q9k7H60XwJRO
eTPcC8RUaMnbovGv39yw0/zfSJja8PpDIFs/7zZT6LoCmZGpbflmItu005kAnUKnoXcLv0aOnk4qTBpC
9qtLpfW4acqzvfVGgAde3zJszYDzHXAzxUMrTVkEJwe6CcJwMH6qWrkj92yLhOJc1qKBygRkB3vpgCZN
iRs55ewjjjHvwPSfgEUnrmoqJpDsmAUUfRadFX5BekN8ovsJWUj+7Pj1m1ZjjGnaGxtF/RimDde2mpJl
XEC1CNtdg1AKovA+Wl608wrPQwGM9xUXp+4FtXCchgq+eD59CBE/9DnCtZvWlsX9mt/QD31QqgXTUHA3
EvpvekMmDzC8rmhL4CCf51c1IugN62Nq5qI9qVSK0tok7iSFBE7eBq559AwF/pTvlbK/ihbhjqwzSN2J
xIUbnLPpnjWznkOpKIrkAvjg6FrhkVZ49IY5D5OSfgJoU06K6UiwXytGEIYcPI55DKhr05vzIfHgUtz4
kejKqxP+jOoF0B6c8gXs9oFD1s+7DBZOitOn2l
Hash da mensagem:
101559824801157566481275910591436486590799948262884771044134935815408357614456
Hash decodificado:
101559824801157566481275910591436486590799948262884771044134935815408357614456
Verificação de assinatura: True
    \end{Verbatim}

    Caso todos os passos tenham sido corretamente executados em sua devida
ordem, a verificação da assinatura terá sido validada com a mensagem:
``Verificação de assinatura: True''. Caso não, execute novamente todos
os blocos desse notebook em sua devida ordem.

Com isto, teremos efetivamente construído um software capaz de gerar
assinaturas digitais, transmiti-las e verificá-las.

    \hypertarget{reposituxf3rio-e-slides-da-apresentauxe7uxe3o}{%
\subsection{Repositório e Slides Da
Apresentação}\label{reposituxf3rio-e-slides-da-apresentauxe7uxe3o}}

Para mais informações sobre a implementação, confira o repositório em:
https://github.com/EduardoMarciano/Seminario-SC

Para mais informações sobre a apresentação, confira os slides em:
https://www.canva.com/design/DAGd5pQ5obo/k4p1LVSKr-lb5PC8-SkCqw/view


    % Add a bibliography block to the postdoc
    
    
    
\end{document}
